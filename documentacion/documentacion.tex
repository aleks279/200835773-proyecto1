\documentclass{article}
\usepackage{graphicx}
\usepackage{fancyhdr}
\usepackage{listings}

\let\<\textless
\let\>\textgreater

\graphicspath{ {images/} }
\pagestyle{fancy}
\fancyhf{}
\rhead{Proyecto \#1}
\rfoot{P\'agina \thepage}

\begin{document}
\begin{titlepage}
  \centering
  {\scshape\LARGE Instituto Tecnol\'ogico de Costa Rica \par}
  \vspace{1cm}
  {\scshape\Large Proyecto \#1\par}
  \vspace{1.5cm}
  {\Large\itshape Sa\'ul Zamora\par}
  \vfill
  profesor\par
  Kevin Moraga \textsc{}

  \vfill

% Bottom of the page
  {\large \today\par}
\end{titlepage}

\section{Introducci\'on}
El sistema se basa en una aplicaci\'on web que permita a sus usuarios comprar y/o vender propiedades de bienes ra\'ices. La meta principal es solventar el problema que normalmente se presenta luego de la compra o venta de una propiedad el cual radica en el traspaso de la misma. Dicha situaci\'on involucra abogados notarios que hagan v\'alida la transacci\'on ante el Registro Nacional.

El sistema pretende, mediante el uso de \emph{blockchain}, eliminar la necesidad del abogado notario, almacenando todas las transacciones de propiedades realizadas en un ``libro p\'ublico'', el cual puede ser accesado por los usuarios y ser\'a validado mediante el m\'etodo de la ``miner\'ia de datos'' que generar\'ia los bloques v\'alidos que constituyen el blockchain.

\section{Ambiente de desarrollo}
\begin{itemize}
  \item Sistema operativo utilizado: Linux Ubuntu 16.04 LTS
  \item Ruby versi\'on 2.3.1
  \item Rails Framework versi\'on 5.0
\end{itemize}

\section{M\'odulos}
\subsection{Users}
Este m\'odulo tendr\'a las funcionalidades respectivas a la creaci\'on, edici\'on y borrado de usuarios. Dentro de los m\'etodos de creaci\'on de usuarios se incluir\'a un login con Facebook.

\subsection{Properties}
En este m\'odulo los usuarios podr\'an agreguar aquellas propiedades que deseen vender. Deber\'an agregar al menos una caracter\'istica y foto a la propiedad, adem\'as de su direcci\'on.

\subsection{Advertisements}
En este m\'odulo los usuarios podr\'an crear, editar o eliminar anuncios de sus propiedades. Dichos anuncios deber\'an contener una descripci\'on breve y una sola propiedad. Cada vez que un anuncio reciba un ``bid'' de parte de otro usuario, el propietario del anuncio recibir\'a una notificaci\'on.

Tambi\'en se podr\'a acceder a los anuncios de otros usuarios, con la opci\'on de poner ``bids'' en ellos.

\subsection{Notifications}
En este m\'odulo, los usuarios podr\'an leer y eliminar las notificaciones que hayan recibido.

\subsection{Bids}
En este m\'odulo, los usuarios podr\'an accesar los ``bids'' que hayan tenido sus anuncions, los cuales tendr\'an informaci\'on de contacto acerca del usuario que puso el ``bid''. Cada vez que un anuncio es eliminado, se eliminan los ``bids'' correspondientes a ese anuncio.

\subsection{Transactions}
En este m\'odulo, los usuarios tendr\'an acceso de lectura al libro p\'ublico con las transacciones realizadas por todos los usuarios del sistema.

\section{Bit\'acora de trabajo}
\begin{itemize}
  \item 14-04-2017:
  \begin{itemize}
    \item 2 horas - dise\~no inicial de la base de datos.
    \item 2 horas - documentaci\'on.
  \end{itemize}
  \item 20-04-2017:
  \begin{itemize}
    \item 1 hora - configuraci\'on b\'asica de la aplicaci\'on de Ruby on Rails usando MySQL y MongoDB.
  \end{itemize}
  \item 25-04-2017:
  \begin{itemize}
    \item 3.5 horas - configuraci\'on inicial de los modelos de base de datos en MySQL y MongoDB. Validaciones b\'asicas en los modelos.
  \end{itemize}
  \item 27-04-2017:
  \begin{itemize}
    \item 2 horas - agragar controladores para Properties y Users. Agregar usuarios administradores. Configuraci\'on de rutas.
  \end{itemize}
\end{itemize}
Total de horas trabajadas: 10.5 horas.

\end{document}